\documentclass{article}
% Dependencias
\usepackage[backend = biber, bibstyle = apa, citestyle = apa, style = apa, sorting = none, sortcites = true, block = none]{biblatex}
\usepackage[utf8]{inputenc}
\usepackage[spanish]{babel}
\usepackage[]{amsthm}
\usepackage{amsmath}
\usepackage[]{amssymb}
\usepackage{graphicx}
\usepackage{wrapfig}
\usepackage[letterpaper, margin=1.5in]{geometry}
\usepackage[hidelinks]{hyperref}
\usepackage{csquotes}
\decimalpoint

% Bibliografía
\makeatletter
\RequireBibliographyStyle{alphabetic}
\makeatother
\addbibresource{references.bib}

\begin{document}
    \begin{titlepage}
        \begin{center}
            \begin{figure}
                \centering
                \includegraphics[scale=0.13]{logo_itesm.png}\\ % Logo de la institución
            \end{figure}
        \vspace{5cm}
        \LARGE{Instituto Tecnológico y de Estudios Superiores de Monterrey}\\
        \fontsize{12}{14}\selectfont
        \vspace{1cm}
        \textbf{Caracterización del entorno de trabajo de un robot explorador en Marte}\\ % Nombre de la tarea
        \vspace{0.7cm}
        % Óscar Antonio Banderas Álvarez \\
        % \vspace{0.2cm}
        % A01568492 \\
        % \vspace{0.2cm}
        % Juan Pablo Echeagaray González\\ % Nombre de autor 1
        % \vspace{0.2cm}
        % A00830646 \\ % Matrícula autor 1
        % \vspace{0.2cm}
        % Erika Martínez Meneses \\
        % \vspace{0.2cm}
        % A01028621 \\
        % \vspace{0.2cm}
        % Emily Rebeca Méndez Cruz\\
        % \vspace{0.2cm}
        % A00830768 \\
        % \vspace{0.2cm}
        % César Guillermo Vázquez Alvarez \\
        % \vspace{0.2cm}
        % A01197857 \\
        % Hay que ver si les gusta la portada con el formato de tabla
        \begin{tabular}{|| c | c ||}
            \hline
            Alumno & Matrícula \\
            \hline
            Óscar Antonio Banderas Álvarez  & A01568492 \\
            \hline
            Juan Pablo Echeagaray González & A00830646 \\
            \hline
            Erika Martínez Meneses & A01028621 \\
            \hline
            Emily Rebeca Méndez Cruz & A00830768 \\
            \hline
            César Guillermo Vázquez Alvarez & A01197857 \\
            \hline
        \end{tabular} \\
        \vspace{0.7cm}
        Diseño de agentes inteligentes\\ % Materia
        \vspace{0.2cm}
        TC2032.101\\ % Clave de la materia
        \vspace{0.2cm}
        Juan Emmanuel Martínez Ledesma \\ % Nombre del profesor
        \vspace{0.7cm}
        27 de febrero del 2022 \\ % Fecha de entrega
        \end{center}
    \end{titlepage}

    \section{Caracterización del entorno de trabajo de un robot explorador en Marte}
        \subsection{Tipos de sensores}
        Sensores de los que dispone el robot:
            \begin{enumerate}
                \item 23 cámaras
                \begin{enumerate}
                    \item Cámara láser
                    \item Cámara panorámica con zoom
                \end{enumerate}
                \item 2 Micrófonos
                \item Infrarrojos
                \item Espectrómetro ultravioleta
                \item Estación meteorológica
                \item Espectrómetro de rayos x para determinar elementos químicos
                \item Radar
            \end{enumerate}

        Propósito de los sensores:
        \begin{enumerate}
            \item 23 cámaras: Durante el descenso, captar imágenes de la superficie de Marte para compararlas con la información en su computadora y corregir la trayectoria en caso necesario. Y una vez en el planeta captar imágenes del planeta a explorar.
            \item 2 Micrófonos: Captar sonidos en el planeta. Uno para grabar sonidos durante el descenso y otro en la superficie.
            \item Infrarrojos: Medir la radiación electromagnética infrarroja de los cuerpos en su campo de visión. Útil para medir la temperatura y detectar objetos calientes, y además permite la visión nocturna y la posibilidad de atravesar algunos objetos opacos para la luz visible.
            \item Espectrómetro ultravioleta: Escanear el terreno y determinar su composición química.
            \item Estación meteorológica: Medir el viento, el polvo, la radiación ultravioleta y otros indicadores del clima del planeta a explorar.
            \item Espectrómetro de rayos x: Escanear el terreno y determinar su composición química.
            \item Radar: Emisión y propagación de ondas electromagnéticas en un medio, con la posterior recepción de las reflexiones que se producen en sus discontinuidades.
        \end{enumerate}

        Información proporcionada:
        \begin{enumerate}
            \item 23 cámaras: Imágenes del planeta a explorar.
            \item 2 Micrófonos: Grabaciones de los sonidos captados durante el descenso y en la superficie del planeta a explorar.
            \item Infrarrojos: Posición de objetos y formas, colores y diferencias de superficie incluso bajo condiciones ambientales extremas.
            \item Espectrómetro ultravioleta: Composición química del terreno.
            \item Estación meteorológica:Indicadores que muestran las condiciones climáticas en las diferentes zonas del planeta que están siendo exploradas.
            \item Espectrómetro de rayos x: Composición química del terreno.
            \item Radar: Cambios en la conductividad, la permitividad eléctrica y la permeabilidad magnética.
        \end{enumerate}

        Consideramos que los sensores más relevantes para que el robot pueda navegar de manera segura son las cámaras ya que con ellas se puede observar la superficie por la que planea avanzar y la superficie en donde planea aterrizar y decidir si es una superficie óptima y proseguir en esa dirección o cambiar el curso por un lado más seguro y la estación meteorológica por el mismo principio, que necesita saber si las condiciones climatológicas de la superficie por la que va a explorar son adecuadas para proseguir sin que se dañe ningún sistema ni parte del robot.

        \subsection{Actuadores}
            \begin{enumerate}
                \item Brazos y manos articuladas: La función de estos es poder recoger cosas, ya sea una piedra para poder analizarla mejor u otras cosas que se podrían encontrar en Marte. 
                \item Ruedas: Su función principal es proporcionar movimiento del robot a lo largo de la superficie de Marte.
            \end{enumerate}

        \subsection{PEAS}

        \subsection{Caracterización del entorno}
            \begin{enumerate}
                \item El ambiente puede ser catalogado como prácticamente completamente observable, ya que podemos hacer que el robot explorador disponga de todos los sensores que necesite para percibir toda la información requerida para su proceso de toma de decisiones
                \item El entorno también puede ser catalogado como de un solo agente, hasta la fecha no hemos encontrado indicios de vida en el planeta rojo
                \item Suponiendo que sea la primera vez que el robot visite Marte, su entorno sería estocástico, ya que hay cierta incertidumbre de como se verá el terreno, pero si el caso de que el robot visitará áreas previamente exploradas, podemos decir que el ambiente es determinista
                \item El entorno en este caso es secuencial, las decisiones del explorador siempre afectarán su entorno y las posibles acciones que pueda tomar en un futuro
                \item (TODO) Pensaría que el ambiente es semi dinámico, le podríamos bajar la medida de desempeño al agente por pasar mucho tiempo deliberando, por el hecho de estar gastando energía sin hacernos algo útil
                \item El ambiente es continuo, se le pedirá al agente que tome decisiones y analice su entorno en todo momento, además de que sus acciones no serán tan simples como \emph{mover - derecha}
                \item El entorno en este caso puede ser visto como conocido, un modelo físico que describa lo que sucedería en el sitio de exploración (dependiendo de las acciones del agente) puede ser provisto con facilidad
            \end{enumerate}

        \subsection{Tipo de agente}

        \subsection{Capacidad de aprender}

    \section{Contribuciones}
        \begin{enumerate}
            \item Óscar Antonio Banderas Álvarez: Actuadores, capacidad de aprender
            \item Juan Pablo Echeagaray González: Caracterización del entorno
            \item Erika Martínez Meneses: Sensores
            \item Emily Rebeca Méndez Cruz: PEAS
            \item César Guillermo Vázquez Álvarez: Tipo de agente
        \end{enumerate}

    \section{Conclusiones}
        \subsection{Óscar Antonio Banderas Álvarez}

        \subsection{Juan Pablo Echeagaray González}

        \subsection{Erika Martínez Meneses}
            En esta actividad podemos observar la inteligencia artificial aplicada en un caso más real y ya no sólo en actividades hipotéticas y con esto observar cómo se comporta y la utilidad que tiene cada componente del robot. Es una forma de ver todavía más claro el tema e identificar las PEAS, el tipo de ambiente, tipo de agente, etc. en un agente real. Me pareció muy interesante el poder analizar cómo funciona un robot de ese estilo y ver como ha avanzado la tecnología, es impresionante ver todo lo que se necesita para explorar otro planeta y como lo han implementado en el robot. Me gustó investigar sobre el perseverance y realizar esta actividad.
            
        \subsection{Emily Rebeca Méndez Cruz}

        \subsection{César Guillermo Vázquez Álvarez}
        
    % Si algo se rompe, probablemente es por la bibliografía
    \nocite{*}
    \printbibliography
\end{document}